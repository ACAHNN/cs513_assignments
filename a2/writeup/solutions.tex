\subsection{Question 1}
Given the matrix 

\begin{eqnarray}
  A = 
  \begin{pmatrix}
    1 & 2 \\
    -1 & 2
  \end{pmatrix}
\end{eqnarray}

\subsubsection{Part A}

The spectrum of \(A\) is found by finding the eigenvalues of \(A\).

\begin{eqnarray}
  A - \lambda I &=& 0 \\
  \begin{bmatrix}
    1 & 2 \\
    -1 & 2
  \end{bmatrix}
  -
  \begin{bmatrix}
    \lambda & 0 \\
    0 & \lambda
  \end{bmatrix}
  &=& 0 \\
  \begin{bmatrix}
    1-\lambda & 2 \\
    -1 & 2-\lambda
  \end{bmatrix}
  &=& 0 \\
  ((1-\lambda)(2-\lambda)) + 2 &=& 0 \\
  \lambda^{2} - 3\lambda + 4 &=& 0
\end{eqnarray}

Now we simply factor to find the eigenvalues.

\begin{eqnarray}
  \lambda &=& \frac{3 \pm \sqrt{-7}}{2} \\
  \lambda &=& \frac{3 \pm \sqrt{7}i}{2}
\end{eqnarray}

Therefore the spectrum of \(A\), 
\begin{eqnarray}
  \boxed{\sigma(A) = \frac{3 \pm \sqrt{7}i}{2}}
\end{eqnarray}

The spectral radius of \(A\) is,

\begin{eqnarray}
  p(A) &=& \max \{\sigma(A)\} \\
  \text{Therefore, using } \\
  \lambda &=& \frac{3 - \sqrt{7}i}{2} \\
  \text{and converting to the reals } \\
  &=& \sqrt{\left(\frac{3 \pm \sqrt{7}i}{2}\right)^2} \\
  &=& \sqrt{\frac{9}{4} + \frac{7}{4}} \\
  &=& \sqrt{\frac{16}{4}} \\
  &=& \sqrt{4} \\
  \boxed{p(A) = 2}
\end{eqnarray}

\subsubsection{Part B}
\begin{eqnarray}
  || A ||_1 = \max \{2,4\} = \boxed{4} \\
  || A ||_{\inf} = \max \{3,3\} = \boxed{3} \\
  || A ||_2 = \sqrt{ \lambda_{\max} (A'A)}
\end{eqnarray}

First we find the spectrum of \(A'A\),

\begin{eqnarray}
  A'A &=&
  \begin{bmatrix}
    1 & -1 \\
    2 & 2
  \end{bmatrix}
  \begin{bmatrix}
    1 & 2 \\
    -1 & 2
  \end{bmatrix}
  \\
  &=&
  \begin{bmatrix}
    2 & 0 \\
    0 & 8
  \end{bmatrix}
\end{eqnarray}

The eigenvalues are simply the diagonal entries meaning,

\begin{eqnarray}
  \sigma(A'A) = \{2,8\} \\
  \text{Therefore, }
  ||A||_2 &=& \sqrt{\max \{2,8\}} \\
  ||A||_2 &=& \boxed{\sqrt{8}}
\end{eqnarray}

\subsubsection{Part C}
The left singular vectors of \(A\) are eigenvectors of \(AA'\); we will denote this \(U\).
The right singular vectors of \(A\) are eigenvectors of \(A'A\); we will dentote this \(V\).
The singular values of \(A\) are the square roots of the non-zero eigenvalues of \(A'A\) and \(AA'\); we will denote this \(S\).

Note,

\begin{eqnarray}
  AA' = 
  \begin{bmatrix}
    5 & 3 \\
    3 & 5
  \end{bmatrix}
  \text{,   }
  A'A = 
  \begin{bmatrix}
    2 & 0 \\
    0 & 8
  \end{bmatrix}
\end{eqnarray}

First, we will find the left singular vectors of \(A\).
The eigenvalues of \(AA'\) are \(\lambda = 2,8\).
We will now find eigenvectors for \(AA'\).

Case: \(\lambda = 2\),
\begin{eqnarray}
  \begin{bmatrix}
    3 & 3 \\
    3 & 3
  \end{bmatrix}
  \begin{bmatrix}
    x_1 \\
    x_2 \\
  \end{bmatrix}
  = 0 \\
  \text{A solution: }
  X = 
  \begin{bmatrix}
    1 \\
    -1
  \end{bmatrix}
\end{eqnarray}

Case: \(\lambda = 8 \),
\begin{eqnarray}
  \begin{bmatrix}
    -3 & 3 \\
    3 & -3
  \end{bmatrix}
  \begin{bmatrix}
    x_1 \\
    x_2 \\
  \end{bmatrix}
  = 0 \\
  \text{A solution: }
  X = 
  \begin{bmatrix}
    1 \\
    1
  \end{bmatrix}
\end{eqnarray}

Therefore,
\begin{eqnarray}
  U = 
  \begin{bmatrix}
    1 & 1 \\
    -1 & 1
  \end{bmatrix}
\end{eqnarray}

Next, we will find the right singular vectors of \(A\).
The eigenvalues of \(A'A\) are \(\lambda = 2,8\)
We will now find eigenvectors for \(A'A\).

Case: \(\lambda = 2\),
\begin{eqnarray}
  \begin{bmatrix}
    0 & 0 \\
    0 & 6
  \end{bmatrix}
  \begin{bmatrix}
    x_1 \\
    x_2 \\
  \end{bmatrix}
  = 0 \\
  \text{A solution: }
  X = 
  \begin{bmatrix}
    1 \\
    0
  \end{bmatrix}
\end{eqnarray}

Case: \(\lambda = 8 \),
\begin{eqnarray}
  \begin{bmatrix}
    -6 & 0 \\
    0 & 0
  \end{bmatrix}
  \begin{bmatrix}
    x_1 \\
    x_2 \\
  \end{bmatrix}
  = 0 \\
  \text{A solution: }
  X = 
  \begin{bmatrix}
    0 \\
    1
  \end{bmatrix}
\end{eqnarray}

Therefore,
\begin{eqnarray}
  V = 
  \begin{bmatrix}
    1 & 0 \\
    0 & 1
  \end{bmatrix}
\end{eqnarray}

The singular values of \(A\) are the square roots of the eigenvalues of \(AA'\) and \(A'A\).
Therefore,
\begin{eqnarray}
  S = 
  \begin{bmatrix}
    \sqrt{2} & 0 \\
    0 & \sqrt{8}
  \end{bmatrix}
\end{eqnarray}

The singular value decomposition of \(A\) is given by \(A = USV'\).

\begin{eqnarray}
  A &=& 
  \begin{bmatrix}
    1 & 1 \\
    -1 & 1
  \end{bmatrix}
  \begin{bmatrix}
    \sqrt{2} & 0 \\
    0 & \sqrt{8}
  \end{bmatrix}
  \begin{bmatrix}
    1 & 0 \\
    0 & 1
  \end{bmatrix}
  \\
  A &=&
  \begin{bmatrix}
    1.4142 & 2.8284 \\
    -1.4142 & 2.8282
  \end{bmatrix}
\end{eqnarray}

The matrix above is in fact not \(A\), but it is only off by a scalar. If we divide by \(1.4142\) we get the original matrix \(A\). Therefore we will encorporate this scalar division into \(U\). Now,

\begin{eqnarray}
  U = 
  \begin{bmatrix}
    0.7071 & 0.7071 \\
    -0.7071 & 0.7071
  \end{bmatrix}
\end{eqnarray}

In full,

\begin{eqnarray}
  U = 
  \begin{bmatrix}
    0.7071 & 0.7071 \\
    -0.7071 & 0.7071
  \end{bmatrix}
  \text{, }
  S = 
  \begin{bmatrix}
    \sqrt{2} & 0 \\
    0 & \sqrt{8}
  \end{bmatrix}
  \text{, }
  V = 
  \begin{bmatrix}
    1 & 0 \\
    0 & 1
  \end{bmatrix}
\end{eqnarray}
\newpage
\subsection{Question 2}
\subsubsection{Part A}
\subsubsection{Part B}
\subsubsection{Part C}
\subsubsection{Part D}
\subsubsection{Part E}

\newpage
\subsection{Question 3}
\subsubsection{Part A}
\subsubsection{Part B}

\newpage
\subsection{Question 4}

\(QR\) factor the matrix \(Z\) where matrix \(Z\) is,

\begin{eqnarray}
  Z = 
  \begin{bmatrix}
    1 & 2 & 3 \\
    4 & 5 & 6 \\
    7 & 8 & 7 \\
    4 & 2 & 3 \\
    4 & 2 & 2
  \end{bmatrix}
\end{eqnarray}
  
This will require three iterations.

Iteration 1:

\begin{eqnarray}
  x &=&
  \begin{bmatrix}
    1 \\
    4 \\
    7 \\
    4 \\
    4
  \end{bmatrix}
  \text{, }
  y = 
  \begin{bmatrix}
    9.8995 \\
    0 \\
    0 \\
    0 \\
    0
  \end{bmatrix}
  \text{, }
  w = 
  \begin{bmatrix}
    -0.6704 \\
    0.3013 \\
    0.5273 \\
    0.3013 \\
    0.3013
  \end{bmatrix}
  \\
  H &=& 
  \begin{bmatrix}
    0.1010 &   0.4041 &   0.7071 &   0.4041 &   0.4041 \\
    0.4041 &   0.8184 &  -0.3178 &  -0.1816 &  -0.1816 \\
    0.7071 &  -0.3178 &   0.4438 &  -0.3178 &  -0.3178 \\
    0.4041 &  -0.1816 &  -0.3178 &   0.8184 &  -0.1816 \\
    0.4041 &  -0.1816 &  -0.3178 &  -0.1816 &   0.8184 \\
  \end{bmatrix}
  \\
  Q &=&
  \begin{bmatrix}
    0.1010 &   0.4041 &   0.7071 &   0.4041 &   0.4041 \\
    0.4041 &   0.8184 &  -0.3178 &  -0.1816 &  -0.1816 \\
    0.7071 &  -0.3178 &   0.4438 &  -0.3178 &  -0.3178 \\
    0.4041 &  -0.1816 &  -0.3178 &   0.8184 &  -0.1816 \\
    0.4041 &  -0.1816 &  -0.3178 &  -0.1816 &   0.8184 \\
  \end{bmatrix}
  \\
  R &=& 
  \begin{bmatrix}
    9.8995 &   9.4954 &   9.6975 \\
   -0.0000 &   1.6311 &   2.9897 \\
   -0.0000 &   2.1044 &   1.7320 \\
   -0.0000 &  -1.3689 &  -0.0103 \\
   -0.0000 &  -1.3689 &  -1.0103 \\
  \end{bmatrix}
\end{eqnarray}

Iteration 2:

\begin{eqnarray}
  x &=&
  \begin{bmatrix}
    9.4954 \\
    1.6311 \\
    2.1044 \\
   -1.3689 \\
   -1.3689 \\
  \end{bmatrix}
  \text{, }
  y = 
  \begin{bmatrix}
    9.4954 \\
    3.2919 \\
         0 \\
         0 \\
         0 \\
  \end{bmatrix}
  \text{, }
  w = 
  \begin{bmatrix}
         0 \\
   -0.5023 \\
    0.6364 \\
   -0.4140 \\
   -0.4140 \\
  \end{bmatrix}
  \\
  H &=& 
  \begin{bmatrix}
    1.0000  &       0 &        0 &        0 &        0 \\
         0  &  0.4955 &   0.6393 &  -0.4158 &  -0.4158 \\
         0  &  0.6393 &   0.1900 &   0.5269 &   0.5269 \\
         0  & -0.4158 &   0.5269 &   0.6572 &  -0.3428 \\
         0  & -0.4158 &   0.5269 &  -0.3428 &   0.6572 \\
  \end{bmatrix}
  \\
  Q &=&
  \begin{bmatrix}
    0.1010 &   0.3162 &   0.8185 &  0.3316 &   0.3316 \\
    0.4041 &   0.3534 &   0.2714 &  -0.5649 &  -0.5649 \\
    0.7071 &   0.3906 &  -0.4537 &  0.2661  &  0.2661 \\
    0.4041 &  -0.5580 &   0.1590 &   0.5082 & -0.4918 \\
    0.4041 &  -0.5580 &   0.1590 &  -0.4918 &   0.5082 \\
  \end{bmatrix}
  \\
  R &=& 
  \begin{bmatrix}
    9.8995 &   9.4954 &   9.6975 \\
   -0.0000 &   3.2919 &   3.0129 \\
   -0.0000 &  -0.0000 &   1.7026 \\ 
   -0.0000 &   0.0000 &   0.0089 \\
   -0.0000 &   0.0000 &  -0.9911 \\
  \end{bmatrix}
\end{eqnarray}

Iteration 3:

\begin{eqnarray}
  x &=&
  \begin{bmatrix}
    9.6975 \\
    3.0129 \\
    1.7026 \\
    0.0089 \\
   -0.9911 \\
  \end{bmatrix}
  \text{, }
  y = 
  \begin{bmatrix}
    9.6975 \\
    3.0129 \\
    1.9701 \\
         0 \\
         0 \\
  \end{bmatrix}
  \text{, }
  w = 
  \begin{bmatrix}
         0 \\
         0 \\
   -0.2606 \\
    0.0086 \\
   -0.9654 \\
  \end{bmatrix}
  \\
  H &=& 
  \begin{bmatrix}
    1.0000 &        0 &        0 &        0 &        0 \\
         0 &   1.0000 &        0 &        0 &        0 \\
         0 &        0 &   0.8642 &   0.0045 &  -0.5031 \\
         0 &        0 &   0.0045 &   0.9999 &   0.0167 \\
         0 &        0 &  -0.5031 &   0.0167 &  -0.8641 \\
  \end{bmatrix}
  \\
  Q &=&
  \begin{bmatrix}
    0.1010 &   0.3162 &   0.5420 &   0.3408 &  -0.6928 \\
    0.4041 &   0.3534 &   0.5162 &  -0.5730 &   0.3422 \\
    0.7071 &   0.3906 &  -0.5248 &   0.2684 &   0.0028 \\
    0.4041 &  -0.5580 &   0.3871 &   0.5006 &   0.3534 \\
    0.4041 &  -0.5580 &  -0.1204 &  -0.4825 &  -0.5273 \\
  \end{bmatrix}
  \\
  R &=& 
  \begin{bmatrix}
    9.8995 &   9.4954 &   9.6975 \\
   -0.0000 &   3.2919 &   3.0129 \\
   -0.0000 &  -0.0000 &   1.9701 \\
   -0.0000 &   0.0000 &  -0.0000 \\
    0.0000 &  -0.0000 &   0.0000 \\
  \end{bmatrix}
\end{eqnarray}

Therefore the \(QR\) factorization of \(Z\) is

\begin{eqnarray}
  Q &=&
  \begin{bmatrix}
    0.1010 &   0.3162 &   0.5420 &   0.3408 &  -0.6928 \\
    0.4041 &   0.3534 &   0.5162 &  -0.5730 &   0.3422 \\
    0.7071 &   0.3906 &  -0.5248 &   0.2684 &   0.0028 \\
    0.4041 &  -0.5580 &   0.3871 &   0.5006 &   0.3534 \\
    0.4041 &  -0.5580 &  -0.1204 &  -0.4825 &  -0.5273 \\
  \end{bmatrix}
  \\
  R &=& 
  \begin{bmatrix}
    9.8995 &   9.4954 &   9.6975 \\
   -0.0000 &   3.2919 &   3.0129 \\
   -0.0000 &  -0.0000 &   1.9701 \\
   -0.0000 &   0.0000 &  -0.0000 \\
    0.0000 &  -0.0000 &   0.0000 \\
  \end{bmatrix}
\end{eqnarray}

\lstinputlisting[caption=Matlab Commands,showstringspaces=false,language=Matlab]{../a2_q4.m}

\newpage
\subsection{Question 5}



%\lstinputlisting[caption=Matlab Commands,showstringspaces=false,language=Matlab]{../a1_q3.m}
