\subsection{Question 1}
Given the matrix 

\begin{eqnarray}
  A = 
  \begin{pmatrix}
    1 & 2 \\
    -1 & 2
  \end{pmatrix}
\end{eqnarray}

\subsubsection{Part A}

The spectrum of \(A\) is found by finding the eigenvalues of \(A\).

\begin{eqnarray}
  A - \lambda I &=& 0 \\
  \begin{bmatrix}
    1 & 2 \\
    -1 & 2
  \end{bmatrix}
  -
  \begin{bmatrix}
    \lambda & 0 \\
    0 & \lambda
  \end{bmatrix}
  &=& 0 \\
  \begin{bmatrix}
    1-\lambda & 2 \\
    -1 & 2-\lambda
  \end{bmatrix}
  &=& 0 \\
  ((1-\lambda)(2-\lambda)) + 2 &=& 0 \\
  \lambda^{2} - 3\lambda + 4 &=& 0
\end{eqnarray}

Now we simply factor to find the eigenvalues.

\begin{eqnarray}
  \lambda &=& \frac{3 \pm \sqrt{-7}}{2} \\
  \lambda &=& \frac{3 \pm \sqrt{7}i}{2}
\end{eqnarray}

Therefore the spectrum of \(A\), 
\begin{eqnarray}
  \boxed{\sigma(A) = \frac{3 \pm \sqrt{7}i}{2}}
\end{eqnarray}

The spectral radius of \(A\) is,

\begin{eqnarray}
  p(A) &=& \max \{\sigma(A)\} \\
  \text{Therefore, using } \\
  \lambda &=& \frac{3 - \sqrt{7}i}{2} \\
  \text{and converting to the reals } \\
  &=& \sqrt{\left(\frac{3 \pm \sqrt{7}i}{2}\right)^2} \\
  &=& \sqrt{\frac{9}{4} + \frac{7}{4}} \\
  &=& \sqrt{\frac{16}{4}} \\
  &=& \sqrt{4} \\
  \boxed{p(A) = 2}
\end{eqnarray}

\lstinputlisting[caption=Matlab Commands,showstringspaces=false,language=Matlab]{../q1_partA}

\subsubsection{Part B}
\begin{eqnarray}
  || A ||_1 = \max \{2,4\} = \boxed{4} \\
  || A ||_{\inf} = \max \{3,3\} = \boxed{3} \\
  || A ||_2 = \sqrt{ \lambda_{\max} (A'A)}
\end{eqnarray}

First we find the spectrum of \(A'A\),

\begin{eqnarray}
  A'A &=&
  \begin{bmatrix}
    1 & -1 \\
    2 & 2
  \end{bmatrix}
  \begin{bmatrix}
    1 & 2 \\
    -1 & 2
  \end{bmatrix}
  \\
  &=&
  \begin{bmatrix}
    2 & 0 \\
    0 & 8
  \end{bmatrix}
\end{eqnarray}

The eigenvalues are simply the diagonal entries meaning,

\begin{eqnarray}
  \sigma(A'A) = \{2,8\} \\
  \text{Therefore, }
  ||A||_2 &=& \sqrt{\max \{2,8\}} \\
  ||A||_2 &=& \boxed{\sqrt{8}}
\end{eqnarray}

\lstinputlisting[caption=Matlab Commands,showstringspaces=false,language=Matlab]{../q1_partB}

\subsubsection{Part C}
The left singular vectors of \(A\) are eigenvectors of \(AA'\); we will denote this \(U\).
The right singular vectors of \(A\) are eigenvectors of \(A'A\); we will dentote this \(V\).
The singular values of \(A\) are the square roots of the non-zero eigenvalues of \(A'A\) and \(AA'\); we will denote this \(S\).

Note,

\begin{eqnarray}
  AA' = 
  \begin{bmatrix}
    5 & 3 \\
    3 & 5
  \end{bmatrix}
  \text{,   }
  A'A = 
  \begin{bmatrix}
    2 & 0 \\
    0 & 8
  \end{bmatrix}
\end{eqnarray}

First, we will find the left singular vectors of \(A\).
The eigenvalues of \(AA'\) are \(\lambda = 2,8\).
We will now find eigenvectors for \(AA'\).

Case: \(\lambda = 2\),
\begin{eqnarray}
  \begin{bmatrix}
    3 & 3 \\
    3 & 3
  \end{bmatrix}
  \begin{bmatrix}
    x_1 \\
    x_2 \\
  \end{bmatrix}
  = 0 \\
  \text{A solution: }
  X = 
  \begin{bmatrix}
    1 \\
    -1
  \end{bmatrix}
\end{eqnarray}

Case: \(\lambda = 8 \),
\begin{eqnarray}
  \begin{bmatrix}
    -3 & 3 \\
    3 & -3
  \end{bmatrix}
  \begin{bmatrix}
    x_1 \\
    x_2 \\
  \end{bmatrix}
  = 0 \\
  \text{A solution: }
  X = 
  \begin{bmatrix}
    1 \\
    1
  \end{bmatrix}
\end{eqnarray}

Therefore,
\begin{eqnarray}
  U = 
  \begin{bmatrix}
    1 & 1 \\
    -1 & 1
  \end{bmatrix}
\end{eqnarray}

Next, we will find the right singular vectors of \(A\).
The eigenvalues of \(A'A\) are \(\lambda = 2,8\)
We will now find eigenvectors for \(A'A\).

Case: \(\lambda = 2\),
\begin{eqnarray}
  \begin{bmatrix}
    0 & 0 \\
    0 & 6
  \end{bmatrix}
  \begin{bmatrix}
    x_1 \\
    x_2 \\
  \end{bmatrix}
  = 0 \\
  \text{A solution: }
  X = 
  \begin{bmatrix}
    1 \\
    0
  \end{bmatrix}
\end{eqnarray}

Case: \(\lambda = 8 \),
\begin{eqnarray}
  \begin{bmatrix}
    -6 & 0 \\
    0 & 0
  \end{bmatrix}
  \begin{bmatrix}
    x_1 \\
    x_2 \\
  \end{bmatrix}
  = 0 \\
  \text{A solution: }
  X =
  \begin{bmatrix}
    0 \\
    1
  \end{bmatrix}
\end{eqnarray}

Therefore,
\begin{eqnarray}
  V = 
  \begin{bmatrix}
    1 & 0 \\
    0 & 1
  \end{bmatrix}
\end{eqnarray}

The singular values of \(A\) are the square roots of the eigenvalues of \(AA'\) and \(A'A\).
Therefore,
\begin{eqnarray}
  S = 
  \begin{bmatrix}
    \sqrt{2} & 0 \\
    0 & \sqrt{8}
  \end{bmatrix}
\end{eqnarray}

The singular value decomposition of \(A\) is given by \(A = USV'\).

\begin{eqnarray}
  A &=& 
  \begin{bmatrix}
    1 & 1 \\
    -1 & 1
  \end{bmatrix}
  \begin{bmatrix}
    \sqrt{2} & 0 \\
    0 & \sqrt{8}
  \end{bmatrix}
  \begin{bmatrix}
    1 & 0 \\
    0 & 1
  \end{bmatrix}
  \\
  A &=&
  \begin{bmatrix}
    1.4142 & 2.8284 \\
    -1.4142 & 2.8282
  \end{bmatrix}
\end{eqnarray}

The matrix above is in fact not \(A\), but it is only off by a scalar. If we divide by \(1.4142\) we get the original matrix \(A\). Therefore we will encorporate this scalar division into \(U\). Now,

\begin{eqnarray}
  U = 
  \begin{bmatrix}
    0.7071 & 0.7071 \\
    -0.7071 & 0.7071
  \end{bmatrix}
\end{eqnarray}

In full,

\begin{eqnarray}
  U = 
  \begin{bmatrix}
    0.7071 & 0.7071 \\
    -0.7071 & 0.7071
  \end{bmatrix}
  \text{, }
  S = 
  \begin{bmatrix}
    \sqrt{2} & 0 \\
    0 & \sqrt{8}
  \end{bmatrix}
  \text{, }
  V = 
  \begin{bmatrix}
    1 & 0 \\
    0 & 1
  \end{bmatrix}
\end{eqnarray}

\lstinputlisting[caption=Matlab Commands,showstringspaces=false,language=Matlab]{../q1_partC}

\newpage
\subsection{Question 2}
\subsubsection{Part A}

The example symmetric matrix is,

\begin{eqnarray}
  A = 
  \begin{bmatrix}
     1 &    2 &    3 &    4 \\
     2 &    5 &    7 &    2 \\
     3 &    7 &    2 &    9 \\
     4 &    2 &    9 &    4 \\    
  \end{bmatrix}
\end{eqnarray}

The eigenvalues and eigenvectors for \(A\) are thus,

\begin{eqnarray}
  \lambda &=&
  \begin{bmatrix}
   17.4770 &        0 &        0 &        0 \\
         0 &  -7.5267 &        0 &        0 \\
         0 &        0 &   3.0000 &        0 \\
         0 &        0 &        0 &  -0.9502 \\
  \end{bmatrix}
  \\
  V &=&
  \begin{bmatrix}
    0.3043 &   0.0760 &   0.2673 &   0.9112 \\
    0.4769 &  -0.3446 &  -0.8018 &   0.1047 \\
    0.6022 &   0.7534 &  -0.0000 &  -0.2640 \\
    0.5633 &  -0.5549 &   0.5345 &  -0.2986 \\
  \end{bmatrix}
\end{eqnarray}

And the eigenvalues and eigenvectors for \(A'A\) are thus,

\begin{eqnarray}
  \lambda &=&
  \begin{bmatrix}
  305.4452 &        0 &        0 &        0 \\
         0 &  56.6519 &        0 &        0 \\
         0 &        0 &   9.0000 &        0 \\
         0 &        0 &        0 &   0.9030 \\
  \end{bmatrix}
  \\
  V &=&
  \begin{bmatrix}
    0.3043 &  -0.0760 &   0.2673 &  -0.9112 \\
    0.4769 &   0.3446 &  -0.8018 &  -0.1047 \\
    0.6022 &  -0.7534 &   0.0000 &   0.2640 \\
    0.5633 &   0.5549 &   0.5345 &   0.2986 \\
  \end{bmatrix}
\end{eqnarray}

Therefore, it is clear from this example that the eigenvectors for \(A\) and \(A'A\) are the same and the eigenvalues of \(A'A\) are the eigenvalues of \(A\) squared.
The conjecture is thus, \((\lambda,v) \text{ is an eigenpair of } A'A \text{ if and only if} (\sqrt{\lambda},v) \text{ is an eigenpair of } A\); (for symmetric \(A\)).

\lstinputlisting[caption=Matlab Commands,showstringspaces=false,language=Matlab]{../q2_partA}

\subsubsection{Part B}

Since \(A\) is a symmetric matrix we know that \(A\) can be decomposed to \(A = PDP^{-1}\). Where the columns of \(P\) are the eigenvectors of \(A\) and \(D\) is a diagonal matrix where the diaganol entries are the eigenvalues of \(A\).
The proof is as follows,

\begin{eqnarray}
  A &=& PDP^{-1} \\
  \text{Squaring both sides, } \\
  A^{2} &=& PDP^{-1}PDP^{-1} \\
  &=& PD(P^{-1}P)P^{-1} \\
  &=& PDDP^{-1} \\
  &=& PD^{2}P^{-1}
\end{eqnarray}

We know that \(A^{2} = A'A\).
The columns of \(P\) are the eigenvectors of \(A'A\) and \(D^{2}\) is a diagonal matrix where the diagonal entries are the eigenvalues of \(A'A\).
Therefore, the eigenvalues of \(A'A\) are the eigenvalues of \(A\) squared.
This concludes the proof.

\subsubsection{Part C}

Theorem: The 2-norm of a symmetric matrix \(A\) is as follows,

\begin{eqnarray}
  ||A||_2 &=& p(A) \\
  \text{Where, }
  p(A) &:=& \max \{|\lambda| : \lambda \in \sigma(A)\}
\end{eqnarray}

We must check the theorm against,

\begin{eqnarray}
  B =
  \begin{bmatrix}
    -92 & 144 \\
    144 & -8 \\
  \end{bmatrix}
\end{eqnarray}

The eigenvalues of \(B\) are \(-200\) and \(100\). Therefore, \(\sigma(B) = \{-200,100\}\) and \(p(B) = 200\).
Using Matlab we find \(||B||_2 = 200\).
Therefore, the theorem holds for \(B\).

\lstinputlisting[caption=Matlab Commands,showstringspaces=false,language=Matlab]{../q2_partC}

\subsubsection{Part D}

This theorem does not hold for the non-symmetric matrix,

\begin{eqnarray}
  C = 
  \begin{bmatrix}
    1 & 2 \\
    3 & 4 \\
  \end{bmatrix}
\end{eqnarray}

We find \(\sigma(C) = \{5.3723, -0.3723\}\) and \(p(C) = 5.3723\).
Using Matlab we find \(||C||_2 = 5.4650\).
Therefore, the theorem does not hold for \(C\).

\lstinputlisting[caption=Matlab Commands,showstringspaces=false,language=Matlab]{../q2_partD}

\subsubsection{Part E}

The singular values of a symmetric matrix \(A\) are the absolute values of the non-zero eigenvalues of \(A\).

\newpage
\subsection{Question 3}
\subsubsection{Part A}

We want to prove that \((\lambda,v)\) is an eigenpair of \(A\) if and only if \((\frac{1}{\lambda},v)\) is an eigenpair of \(A^{-1}\).
The proof is as follows,

\begin{eqnarray}
  Av &=& \lambda v \\
  A^{-1}Av &=& A^{-1}\lambda v \\
  (A^{-1}A)v &=& A^{-1}\lambda v \\
  Iv &=& A^{-1}\lambda v \\
  v &=& A^{-1}\lambda v \\
  \frac{1}{\lambda}v &=& A^{-1}v
\end{eqnarray}

The 'only if' portion follows by symmetry.

\subsubsection{Part B}

The 2-norm of \(A^{-1}\) should be 1 / ( the smallest singular value of \(A\)).
We will denote the singular values of \(A\) as \(s(A)\).
Stated formally,

\begin{eqnarray}
  ||A^{-1}||_2 = \max_{s_A \in s(A)} \{\frac{1}{s_a}\}
\end{eqnarray}

This is true because we know from (a) that \((\lambda,v)\) is an eigenpair for \(A\) if and only if \((\frac{1}{\lambda},v)\) is an eigenpair of \(A^{-1}\).
Therefore, it must be the case that \((\lambda,v)\) is an eigenpair for \(A'A\) if and only if \((\frac{1}{\lambda},v)\) is an eigenpair of \(({A^{-1}})'A^{-1}\) because the 2-norm is preserved.
Since the \(\lambda\)'s of \(A'A\) are the singular values of \(A\) it follows that \(\frac{1}{\lambda}\)'s of \(A'A\) are the singular values of \(A^{-1}\).
Therefore, since \(||A||_2 = \max \{s(A)\}\),

\begin{eqnarray}
  ||A^{-1}||_2 = \max_{s_A \in s(A)} \{\frac{1}{s_a}\}
\end{eqnarray}




\newpage
\subsection{Question 4}

\(QR\) factor the matrix \(Z\) where matrix \(Z\) is,

\begin{eqnarray}
  Z = 
  \begin{bmatrix}
    1 & 2 & 3 \\
    4 & 5 & 6 \\
    7 & 8 & 7 \\
    4 & 2 & 3 \\
    4 & 2 & 2
  \end{bmatrix}
\end{eqnarray}
  
This will require three iterations.

Iteration 1:

\begin{eqnarray}
  x &=&
  \begin{bmatrix}
    1 \\
    4 \\
    7 \\
    4 \\
    4
  \end{bmatrix}
  \text{, }
  y = 
  \begin{bmatrix}
    9.8995 \\
    0 \\
    0 \\
    0 \\
    0
  \end{bmatrix}
  \text{, }
  w = 
  \begin{bmatrix}
    -0.6704 \\
    0.3013 \\
    0.5273 \\
    0.3013 \\
    0.3013
  \end{bmatrix}
  \\
  H &=& 
  \begin{bmatrix}
    0.1010 &   0.4041 &   0.7071 &   0.4041 &   0.4041 \\
    0.4041 &   0.8184 &  -0.3178 &  -0.1816 &  -0.1816 \\
    0.7071 &  -0.3178 &   0.4438 &  -0.3178 &  -0.3178 \\
    0.4041 &  -0.1816 &  -0.3178 &   0.8184 &  -0.1816 \\
    0.4041 &  -0.1816 &  -0.3178 &  -0.1816 &   0.8184 \\
  \end{bmatrix}
  \\
  Q &=&
  \begin{bmatrix}
    0.1010 &   0.4041 &   0.7071 &   0.4041 &   0.4041 \\
    0.4041 &   0.8184 &  -0.3178 &  -0.1816 &  -0.1816 \\
    0.7071 &  -0.3178 &   0.4438 &  -0.3178 &  -0.3178 \\
    0.4041 &  -0.1816 &  -0.3178 &   0.8184 &  -0.1816 \\
    0.4041 &  -0.1816 &  -0.3178 &  -0.1816 &   0.8184 \\
  \end{bmatrix}
  \\
  R &=& 
  \begin{bmatrix}
    9.8995 &   9.4954 &   9.6975 \\
   -0.0000 &   1.6311 &   2.9897 \\
   -0.0000 &   2.1044 &   1.7320 \\
   -0.0000 &  -1.3689 &  -0.0103 \\
   -0.0000 &  -1.3689 &  -1.0103 \\
  \end{bmatrix}
\end{eqnarray}

Iteration 2:

\begin{eqnarray}
  x &=&
  \begin{bmatrix}
    9.4954 \\
    1.6311 \\
    2.1044 \\
   -1.3689 \\
   -1.3689 \\
  \end{bmatrix}
  \text{, }
  y = 
  \begin{bmatrix}
    9.4954 \\
    3.2919 \\
         0 \\
         0 \\
         0 \\
  \end{bmatrix}
  \text{, }
  w = 
  \begin{bmatrix}
         0 \\
   -0.5023 \\
    0.6364 \\
   -0.4140 \\
   -0.4140 \\
  \end{bmatrix}
  \\
  H &=& 
  \begin{bmatrix}
    1.0000  &       0 &        0 &        0 &        0 \\
         0  &  0.4955 &   0.6393 &  -0.4158 &  -0.4158 \\
         0  &  0.6393 &   0.1900 &   0.5269 &   0.5269 \\
         0  & -0.4158 &   0.5269 &   0.6572 &  -0.3428 \\
         0  & -0.4158 &   0.5269 &  -0.3428 &   0.6572 \\
  \end{bmatrix}
  \\
  Q &=&
  \begin{bmatrix}
    0.1010 &   0.3162 &   0.8185 &  0.3316 &   0.3316 \\
    0.4041 &   0.3534 &   0.2714 &  -0.5649 &  -0.5649 \\
    0.7071 &   0.3906 &  -0.4537 &  0.2661  &  0.2661 \\
    0.4041 &  -0.5580 &   0.1590 &   0.5082 & -0.4918 \\
    0.4041 &  -0.5580 &   0.1590 &  -0.4918 &   0.5082 \\
  \end{bmatrix}
  \\
  R &=& 
  \begin{bmatrix}
    9.8995 &   9.4954 &   9.6975 \\
   -0.0000 &   3.2919 &   3.0129 \\
   -0.0000 &  -0.0000 &   1.7026 \\ 
   -0.0000 &   0.0000 &   0.0089 \\
   -0.0000 &   0.0000 &  -0.9911 \\
  \end{bmatrix}
\end{eqnarray}

Iteration 3:

\begin{eqnarray}
  x &=&
  \begin{bmatrix}
    9.6975 \\
    3.0129 \\
    1.7026 \\
    0.0089 \\
   -0.9911 \\
  \end{bmatrix}
  \text{, }
  y = 
  \begin{bmatrix}
    9.6975 \\
    3.0129 \\
    1.9701 \\
         0 \\
         0 \\
  \end{bmatrix}
  \text{, }
  w = 
  \begin{bmatrix}
         0 \\
         0 \\
   -0.2606 \\
    0.0086 \\
   -0.9654 \\
  \end{bmatrix}
  \\
  H &=& 
  \begin{bmatrix}
    1.0000 &        0 &        0 &        0 &        0 \\
         0 &   1.0000 &        0 &        0 &        0 \\
         0 &        0 &   0.8642 &   0.0045 &  -0.5031 \\
         0 &        0 &   0.0045 &   0.9999 &   0.0167 \\
         0 &        0 &  -0.5031 &   0.0167 &  -0.8641 \\
  \end{bmatrix}
  \\
  Q &=&
  \begin{bmatrix}
    0.1010 &   0.3162 &   0.5420 &   0.3408 &  -0.6928 \\
    0.4041 &   0.3534 &   0.5162 &  -0.5730 &   0.3422 \\
    0.7071 &   0.3906 &  -0.5248 &   0.2684 &   0.0028 \\
    0.4041 &  -0.5580 &   0.3871 &   0.5006 &   0.3534 \\
    0.4041 &  -0.5580 &  -0.1204 &  -0.4825 &  -0.5273 \\
  \end{bmatrix}
  \\
  R &=& 
  \begin{bmatrix}
    9.8995 &   9.4954 &   9.6975 \\
   -0.0000 &   3.2919 &   3.0129 \\
   -0.0000 &  -0.0000 &   1.9701 \\
   -0.0000 &   0.0000 &  -0.0000 \\
    0.0000 &  -0.0000 &   0.0000 \\
  \end{bmatrix}
\end{eqnarray}

Therefore the \(QR\) factorization of \(Z\) is

\begin{eqnarray}
  Q &=&
  \begin{bmatrix}
    0.1010 &   0.3162 &   0.5420 &   0.3408 &  -0.6928 \\
    0.4041 &   0.3534 &   0.5162 &  -0.5730 &   0.3422 \\
    0.7071 &   0.3906 &  -0.5248 &   0.2684 &   0.0028 \\
    0.4041 &  -0.5580 &   0.3871 &   0.5006 &   0.3534 \\
    0.4041 &  -0.5580 &  -0.1204 &  -0.4825 &  -0.5273 \\
  \end{bmatrix}
  \\
  R &=& 
  \begin{bmatrix}
    9.8995 &   9.4954 &   9.6975 \\
   -0.0000 &   3.2919 &   3.0129 \\
   -0.0000 &  -0.0000 &   1.9701 \\
   -0.0000 &   0.0000 &  -0.0000 \\
    0.0000 &  -0.0000 &   0.0000 \\
  \end{bmatrix}
\end{eqnarray}

\newpage
\lstinputlisting[caption=Matlab Commands,showstringspaces=false,language=Matlab]{../a2_q4.m}

\newpage
\subsection{Question 5}

\subsubsection{Part A}

The formula to compute a Householder matrix \(H_i\) is \(H_i = I - 2ww'\).
\(H_i\) is then multiplied by a matrix \(A_i\) to generate \(A_{i+1}\).
The problem is the computation of \(A_{i+1}\) is \(O(m^{3})\).
Therefore, we want to compute \(A_{i+1} = H_i A_i\) without explicitly calculating \(H_i\) first.
This can be done by utilizing the formula for computing \(H_i\).

\begin{eqnarray}
  A_{i+1} &=& H_i A_i \\
  &=& (I - 2ww')A_i \\
  &=& I A_i  - 2ww'A_i \\
  &=& A_i - 2ww' A_i \\
  &=& \boxed{A_i - (2w)(w'A_i)}
\end{eqnarray}

This requires \(O(m^{2}\) operations because it only multiplies a matrix by a vector and a vector by a vector.
Note, \(IA_i = A_i\) so this matrix multiplication is done implicitly and not explicitly.
The intuition is made explicit below,

\begin{eqnarray}
  A_{i+1} &=& \underbrace{A_i}_{\text{mxm}} - \underbrace{(2w)}_{\text{mx1}}\underbrace{(w'A_i)}_{\text{1xn}}
\end{eqnarray}

The complexity of each operation in isolation is

\begin{eqnarray}
  IA_i \Rightarrow O(1) \\
  2w \Rightarrow O(m) \\
  w'A_i \Rightarrow O(m^{2}) \\
  (2w)(w'A_i) \Rightarrow O(m^{2}) \\
  A_i - ((2w)(w'A_i)) \Rightarrow O(m)
\end{eqnarray}

Therefore the complexity is dominated by the \(O(m^{2})\) terms and the complexity of the operation is \(O(m^{2})\).

\newpage
\subsubsection{Part B}
\lstinputlisting[caption=Matlab Commands,showstringspaces=false,language=Matlab]{../linear_qr.m}

\subsubsection{Part C}

\lstinputlisting[caption=Matlab Commands,showstringspaces=false,language=Matlab]{../q5_partC}

\subsubsection{Part D}

For a matrix \(A_{mxn}\), if we add the complexity of each line of the algorithm from (c) we have,

\begin{eqnarray}
  m+m+2m+2m^{2}+2m+2m^{2}+2m+m+m^{2}+m^{2} \\
  \boxed{6m^{2}+9m}
\end{eqnarray}

Therefore, the complexity of the algorithm is \(O(m^{2})\).
